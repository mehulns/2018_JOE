%%*************************************************************************
\section{Conclusions}

This study presents a novel, fully automated method to identify safe landing sites on the based on mm-resolution seafloor bathymetry. The conditions that need to be satisfied for safe landing considering the seafloor terrain and vehicle geometry have been derived and these are used to allow an autonomous platform to identify and select landing sites based on a cost function. The method is applied to seafloor data acquired using high resolution laser bathymetry along a transect of $500$\,m, successfully identifying and selecting landing coordinates and headings. The computational times demonstrate that the calculations are applicable to in mission operations from an AUV and that the approach can enable autonomous platforms to use in-situ sensors that require a vehicle to have landed for their operations. Vehicle hardware and seafloor mapping system considerations specific to autonomous landing are described and the concept of a negatively buoyant, fast transiting landing vehicle are presented.

%%*************************************************************************

\section*{Acknowledgments}

The authors would like to thank Y. Nishida of Kyushu Institute of Technology, K. Nagahashi and K. Nagano of Mitsui Engineering and Shipbuilding, U. Neettiyath of The University of Tokyo and T. Koike of Kaiyo Engineering for their help in deployment of BOSS-A during the KR$16-01$ cruise of R/V Kairei. This work was funded under the Program for the Development of Fundamental Tools for the Utilization of Marine Resources of the Japanese Ministry of Education.
