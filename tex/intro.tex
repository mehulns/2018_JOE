\section{Introduction}

\IEEEPARstart{T}he use of unmanned underwater vehicles for exploration of mineral deposits \cite{Suzuki2015}, such as manganese crusts \cite{Usui2016}, manganese nodules \cite{Schoening2016} and seafloor massive sulfides \cite{Thornton2016} has gained momentum in recent years. While high resolution bathymetric maps of the seafloor generated using acoustic \cite{Moustier1993} or visual \cite{Roman2010} mapping systems are useful for recognizing visible and morphological seafloor features \cite{Pizarro2009}\cite{Maki2010}, measurement of chemical composition \cite{Takahashi2015} or frictional coefficient \cite{White2011} of seafloor deposits require direct contact for obtaining measurements. There have been significant developments over the past decade that provide in-situ methods to make measurements of the chemical and geological properties of the seafloor, such as underwater microscopy \cite{Rubin2007}, gamma radiation measurements \cite{Thornton2013a}\cite{Thornton2013}, laser induced breakdown spectroscopy (LIBS) \cite{Thornton2015}\cite{Saeki2014}, laser Raman spectroscopy \cite{Brewer2004}\cite{Pasteris2004} and seafloor stiffness and frictional coefficients \cite{Stanier2015}. The development of these new classes of analytical sensors that requires direct contact motivates the development of landing capabilities for Autonomous Underwater Vehicles (AUVs) to deliver these abilities in a more scalable manner. Underwater terrain in regions such as those with mineral deposits can change abruptly and vary at short intervals on spatial scales that cannot be observed from the surface. Therefore the reliable use of in-situ instruments such as those described, and the safely of the underwater vehicle requires real-time detection of safe landing sites. Although ROV pilots can identify safe landing sites to achieve the proximity needed to perform in-situ chemical measurements with the instrument, at present AUVs lack this capability. 

In this research, we developed a framework to enable an underwater vehicle land on the seafloor autonomously by identifying safe landing sites in real-time. 
	The design of an underwater vehicle was proposed by identifying hardware
	suitable for landing on the seafloor. A mm-resolution mapping system using 
	structured light was also implemented as part of the vehicle design. The 
	conditions for safe landing were identified to develop an algorithm which 
	uses this mm-resolution seafloor bathymetry to identify landing area on 
	the seafloor and an exclusion zone where the centre of gravity is 
	prohibited from landing. Within this landing area, the algorithm detects 
	safe landing sites along different landing headings where an underwater 
	vehicle of a given geometry can fit. The algorithm selects the final landing 
	site from the candidates using a cost function that takes into account 
	properties of the seafloor terrain at the sites. The algorithm was 
	implemented on over $1000$ sq.m. of seafloor bathymetry obtained by an AUV 
	with an equivalent mapping system at the No.$5$ Takuyo seamount in the 
	Northwest-Pacific to evaluate its performance. The results are analyzed to 
	demonstrate the feasibility of using the landing framework in real seafloor 
	surveys.
	
The remainder of this paper is organized as follows; section II discusses the challenges associated with autonomous landing of an underwater vehicle along with previous research in this field. Section II describes the proposed design of an underwater vehicle capable of landing. The high resolution mapping system for generating bathymetry with mm-resolution is also described. In  Section III, the different steps of the algorithm to identify landing sites are described and demonstrated by simulating its performance on seafloor data obtained using an equivalent high resolution mapping system.  Section IV describes the implementation of the algorithm on seafloor bathymetry obtained using an AUV during a real underwater survey, the results of which are also published. Section V provides conclusion to this work. 
