\section{Introduction}

\IEEEPARstart{T}he use of unmanned underwater vehicles for exploration of mineral deposits \cite{Suzuki2015}, such as manganese crusts \cite{Usui2016}, manganese nodules \cite{Schoening2016} and seafloor massive sulfides \cite{Thornton2016} has gained momentum in recent years. While high resolution bathymetric maps of the seafloor generated using acoustic \cite{Moustier1993} or visual \cite{Roman2010} mapping systems are useful for recognizing visible and morphological seafloor features \cite{Pizarro2009,Maki2010}, measurement of chemical composition \cite{Takahashi2015} or frictional coefficient \cite{White2011} of seafloor deposits require direct contact for obtaining measurements. There have been significant developments over the past decade that provide in-situ methods to make measurements of the chemical and geological properties of the seafloor, such as underwater microscopy \cite{Rubin2007}, gamma radiation measurements \cite{Thornton2013a,Thornton2013}, laser induced breakdown spectroscopy (LIBS) \cite{Thornton2015}, laser Raman spectroscopy \cite{Brewer2004,Pasteris2004} and seafloor stiffness and frictional coefficients \cite{Stanier2015}. The development of these new classes of analytical sensors that requires direct contact motivates the development of landing capabilities for Autonomous Underwater Vehicles (AUVs) to deliver these capabilities in a more scalable manner. The underwater terrain however, can change abruptly on spatial scales that cannot be observed from the surface. Therefore the reliable use of in-situ instruments such as those described, and the safety of the underwater vehicle requires real-time detection of suitable landing sites. Although remotely operated vehicle (ROV) pilots routinely identify safe landing sites and perform manipulations or in-situ chemical measurements with the instruments described, AUVs currently lack the sensing and data processing capabilities needed for these tasks. 

In this research, we developed a framework to enable an underwater vehicle to autonomously identify safe landing sites based on in-situ measurements. The design concept of an underwater vehicle is proposed, identifying the hardware modifications needed for landing of AUVs on the seafloor. This includes a landing skid, and a mm-resolution laser mapping system used to detect safe landing areas. The conditions for safe landing are identified and used to develop an algorithm that uses the mm-resolution bathymetry information to identify landing areas considering the geometry and righting moment of the AUV. The algorithm detects safe landing sites along different headings, selecting the most suitable landing site based on a cost function that takes into account the slope and rugosity of the seafloor. The performance of the algorithm is verified by analysing more than 1000\,m$^2$ of mm-resolution seafloor bathymetry obtained by an AUV along a 500\,m transect on the slopes of the Takuyo Daigo seamount (located in the Northwest-Pacific) at an average depth of 1400\,m. The results demonstrate the feasibility of safe autonomous landing in real seafloor terrains.